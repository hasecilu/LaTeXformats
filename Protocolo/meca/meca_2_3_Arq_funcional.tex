\section{Arquitectura funcional}
\label{Arquitectura_funcional}

Arquitectura funcional
\par
\section{Interconexi\'on de funciones}
%
A continuaci\'on se muestra el desarrollo del modelo de interconecci\'on entre funciones IDEF-0. Se incluye hasta el nivel 2 de las funciones establecidas en el modelo FBS. Citar IDEF 0. En la Figura 2 se observa las relaciones entre la funciones en su primer nivel. La asignaci\'on entre funci\'on y proceso se realiz\'o de la manera siguiente, F1:A1, F2:A2, F3:A3, F4:A4.

Para definir \textbf{Manipular posici\'on de componentes} (A1), se considera como entrada la Posici\'on a manipular, es decir, la posici\'on desde la cual el robot iniciar\'a el movimiento para recoger o colocar alg\'un componente. En cuanto a las variables de control, se consideran a los par\'ametros de manipulaci\'on (fuerza de manipulaci\'on, velocidad del movimiento y aceleraci\'on del movimiento) y a la posici\'on requerida. Los recursos que necesita el proceso para cumplir la funci\'on son, los componentes a manipular, los dispositivos de manipulaci\'on (dispositivos de recolecci\'on y colocaci\'on, dispositivos de generaci\'on de movimiento y dispositivos de acoplamiento de movimiento) y la energ\'ia el\'ectrica con la que alimentar\'a a los dispositivos. Como salida de este proceso se tiene la posici\'on de los componentes manipulada, es decir la posici\'on en "tiempo real", y esta ser\'a un par\'ametro de control en el proceso de Gesti\'on de informaci\'on.

El proceso \textbf{Gestionar energ\'ia} (A2) se describe a continuaci\'on. Como entrada, se requiere energ\'ia el\'ectrica, todav\'ia no est\'a caracterizada, pero es la energ\'ia con la que se alimentar\'a a todo el sistema. Como control, se tiene a los par\'ametros de la energ\'ia, con esto se entiende a los cambios necesarios que se realizar\'an en la energ\'ia de entrada, tanto de potencia el\'ectrica, como a los de frecuencia, de acuerdo a los requisitos de cada m\'odulo. Los recursos que se necesitar\'an este proceso se nombran dispositivos de gesti\'on de energ\'ia; estos son los dispositivos que se encargar\'an de acondicionar, almacenar, distribuir y disipar. Como salida de este proceso tambi\'en est\'a energ\'ia el\'ectrica, pero con las condiciones necesarias para poder abastecer a los procesos que la requieran.

La \textbf{Gesti\'on de informaci\'on} (A3) recibir\'a como entrada instrucciones externas tanto de alg\'un usuario, como de alg\'un otro sistema. Para controlar el proceso se requiere a la posici\'on manipulada en "tiempo real" y a los par\'ametros de informaci\'on, es decir, las caracter\'isticas de los datos que se deben enviar y recibir (nivel l\'ogico, bits por segundo, etc), y a el medio mediante el cual se realizar\'a la comunicaci\'on de estos (comunicaci\'on serial, paralela, etc). Para poder cumplir con este proceso se requiere energ\'ia el\'ectrica, a un usuario, y a dispositivos de gesti\'on de informaci\'on. Estos incluyen, dispositivos  para acondicionar, procesar y comunicar informaci\'on; sensores para medir posici\'on, rotaci\'on, velocidad, aceleraci\'on, fuerza y temperatura; y a un dispositivo embebido al que se cargar\'an los algoritmos de control para tomar las decisiones. De esta forma, a la salida se obtendr\'a energ\'ia gestionada que se distribuir\'a a cada uno de los m\'odulos que la necesiten con las condiciones requeridas.

El proceso encargado de \textbf{Soportar y proteger} (A4) suministrar\'a a todos los dem\'as m\'odulos despu\'es de haberles condiciones de protecci\'on para evitar que se da\~{n}e al usuario, a los componentes y PCB, al robot mismo. Es por esto que los recursos necesarios son los dispositivos a proteger y soportar; los dispositivos de protecci\'on y soporte; y la energ\'ia que alimentar\'a a los dispositivos de protecci\'on. Para controlar esta funci\'on se necesitan los par\'ametros de protecci\'on que incluye a las cargas externas (mec\'anicas, el\'ectricas, t\'ermicas, de software) que deber\'a soportar; y monitoreo del "estado actual" del proceso en cada de ser necesario su bloqueo.

En las Figuras 3 y 4 se muestra el Nivel dos del diagrama de interconexiones IDEF-0.
\newline\newline\newline\newline\newline\newline\newline\newline\newline\newline\newline
\newline\newline\newline\newline\newline\newline\newline\newline\newline\newline\newline
\newline\newline\newline\newline\newline\newline\newline\newline\newline\newline\newline
\newline\newline\newline\newline\newline\newline\newline\newline\newline\newline\newline
\newline\newline\newline\newline\newline\newline\newline\newline\newline\newline\newline
\newline\newline\newline\newline\newline\newline\newline\newline\newline\newline\newline


\begin{figure}[hbtp]
\includegraphics[angle=90,scale=0.4]{images/IDEF0_N1.jpg}
\caption{IDEF0 Nivel 1}
\end{figure}

\begin{figure}[hbtp]
\includegraphics[angle=90,scale=0.5]{images/IDEF0_N2_A1.jpg}
\caption{IDEF0 Nivel 2 Manipular posici\'on de componentes}
\end{figure}

\begin{figure}[hbtp]
\includegraphics[angle=90,scale=0.4]{images/IDEF0_N2_A2.jpg}
\caption{IDEF0 Nivel 2 Gesti\'on de energ\'ia}
\end{figure}

