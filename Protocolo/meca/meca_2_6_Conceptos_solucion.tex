\section{Generaci\'on de conceptos soluci\'on}
\label{Generacion_conceptos_solucion}

Generaci\'on de conceptos soluci\'on
\par

\textbf{M\'odulo energ\'etico}

El \textit{m\'odulo energ\'etico} (M1) pertenece al \textit{sistema estructural} (S1), las funciones asociadas a este m\'odulo son las siguientes: F\textsubscript{2}, F\textsubscript{2-1}, F\textsubscript{2-2}, F\textsubscript{2-3} y F\textsubscript{2-4}.

\begin{center}
\begin{adjustbox}{angle=90}
\begin{tabular}{|l|c|c|c|c|}
\hline
\textbf{Caracter\'isticas} &
 \textbf{Soluci\'on 1} &
 \textbf{Soluci\'on 2} &
 \textbf{Soluci\'on 3} &
 \textbf{Soluci\'on 4}\\\hline
C\textsubscript{1-1}.- Fuente de energ\'ia &
 Fuente lineal &
 Fuente conmutada &
   &
 \\\hline
C\textsubscript{1-2}.- Regulaci\'on de energ\'ia &
 Salida fija &
 Salida variable &
   &
 \\\hline
C\textsubscript{1-3}.- Almacenamiento de energ\'ia &
 Interno &
 Externo\newline &
   &
 \\\hline
C\textsubscript{1-4}.- Disipaci\'on de calor &
 Con aire &
 Fluido l\'iquido &
 Disipador met\'alico &
 Disipador met\'alico y aire\\\hline
\end{tabular}
\end{adjustbox}
\end{center}

\par Se generaron 32 combinaciones.


%%%%%%%%%%%%%%%%%%%%%%%%%%%%%%%%%%%%%%%%%%%%%%%%%%
\textbf{M\'odulo de medici\'on de par\'ametros}

El m\'odulo d\textit{e medici\'on de par\'ametros} (M2) pertenece al \textit{sistema de procesamiento central} (S2), las funciones asociadas a este m\'odulo son las siguientes: F\textsubscript{3}, F\textsubscript{3-1}, F\textsubscript{3-1-1}, F\textsubscript{3-1-1-1}, F\textsubscript{3-1-1-2}, F\textsubscript{3-1-1-2-1}, F\textsubscript{3-1-1-2-2}, F\textsubscript{3-1-1-3}, F\textsubscript{3-1-1-4}, F\textsubscript{3-1-1-5}, F\textsubscript{3-1-2}, F\textsubscript{3-1-2-1}, F\textsubscript{3-1-2-2}, F\textsubscript{3-1-2-2-1}, F\textsubscript{3-1-2-2-2}.

\begin{center}
\begin{adjustbox}{angle=90}
\begin{tabular}{|l|c|c|c|c|}
\hline
\textbf{Caracter\'isticas} &
 \textbf{Soluci\'on 1} &
 \textbf{Soluci\'on 2} &
 \textbf{Soluci\'on 3} &
 \textbf{Soluci\'on 4}\\\hline
C\textsubscript{2-1}.- Sensor de posici\'on angular &
 Anal\'ogico &
 Digital &
   &
 \\\hline
C\textsubscript{2-2}.- Sensor de corriente &
 Invasivo &
 No invasivo &
   &
 \\\hline
C\textsubscript{2-3}.- Posici\'on sensor de posici\'on angular &
 Eje delantero &
 Eje trasero &
 Externo &
 \\\hline
C\textsubscript{2-4}.- Sensores de l\'imite de carrera &
 \'Opticos &
 Mec\'anicos &
 Mang\'eticos &
 \\\hline
\end{tabular}
\end{adjustbox}
\end{center}

\par Se generaron 36 combinaciones.\par
Hay 1152 combinaciones acumuladas.

%%%%%%%%%%%%%%%%%%%%%%%%%%%%%%%%%%%%%%%%%%%%%%%%%%
\textbf{M\'odulo de procesamiento de informaci\'on}

El \textit{m\'odulo de procesamiento de informaci\'on} (M5) pertenece al \textit{sistema de procesamiento central} (S2), las funciones asociadas a este m\'odulo son las siguientes: F\textsubscript{3}, F\textsubscript{3-2}, F\textsubscript{3-3}, F\textsubscript{3-4}, F\textsubscript{3-5}, F\textsubscript{3-5-1}, F\textsubscript{3-5-2}, F\textsubscript{3-5-3}, F\textsubscript{3-5-4}, F\textsubscript{3-5-5}, F\textsubscript{3-5-6}, F\textsubscript{3-5-7}, F\textsubscript{3-6}, F\textsubscript{3-6-1}, F\textsubscript{3-6-2}, F\textsubscript{3-6-2-1}, F\textsubscript{3-6-2-2}

\begin{center}
\begin{adjustbox}{angle=90}
\begin{tabular}{|l|c|c|c|c|}
\hline
\textbf{Caracter\'isticas} &
 \textbf{Soluci\'on 1} &
 \textbf{Soluci\'on 2} &
 \textbf{Soluci\'on 3} &
 \textbf{Soluci\'on 4}\\\hline
C\textsubscript{5-1}.- Dispositivo de procesamiento &
 Microprocesador &
 Microcontrolador &
 FPGA &
 Single-Board Computer\\\hline
\end{tabular}

\end{adjustbox}
\end{center}

Se generaron 4 combinaciones.\par
\par Hay 4,608 combinaciones acumuladas.

%%%%%%%%%%%%%%%%%%%%%%%%%%%%%%%%%%%%%%%%%%%%%%%%%%
\textbf{M\'odulo de movimiento}

El \textit{m\'odulo de movimiento} (M3) pertenece al \textit{sistema de manipulaci\'on} (S3), las funciones asociadas a este m\'odulo son las siguientes: F\textsubscript{1}, F\textsubscript{1-1}, F\textsubscript{1-1-1}, F\textsubscript{1-1-2}, F\textsubscript{1-2}, F\textsubscript{1-3}, F\textsubscript{1-3-1}, F\textsubscript{1-3-2}, F\textsubscript{1-3-2-1}, F\textsubscript{1-3-2-2}, F\textsubscript{1-3-3}.


\begin{center}
\begin{adjustbox}{angle=90}
\begin{tabular}{|l|c|c|c|c|}
\hline
\textbf{Caracter\'isticas} &
 \textbf{Soluci\'on 1} &
 \textbf{Soluci\'on 2} &
 \textbf{Soluci\'on 3} &
 \textbf{Soluci\'on 4}\\\hline
C\textsubscript{3-1}.- Morfolog\'ia de robot &
 SCARA &
 Cartesiano &
 Delta &
 0 \\\hline
C\textsubscript{3-2}.- Tipo de motor &
 Motor CD &
 Motor AC &
 Servomotor &
 Motor a pasos\\\hline
C\textsubscript{3-3}.- Tipo de transmisi\'on &
 Banda &
 Tornillo &
 Cadena &
 Engranes\\\hline
\end{tabular}

\end{adjustbox}
\end{center}

Se generaron 48 combinaciones.\par
\par Hay 221,184 combinaciones acumuladas.

%%%%%%%%%%%%%%%%%%%%%%%%%%%%%%%%%%%%%%%%%%%%%%%%%%
\textbf{M\'odulo de agarre}

El \textit{m\'odulo de agarre} (M4) pertenece al \textit{sistema de manipulaci\'on} (S3), las funciones asociadas a este m\'odulo son las siguientes: F\textsubscript{1}
, F\textsubscript{1-1}, F\textsubscript{1-1-1}, F\textsubscript{1-1-2}, F\textsubscript{1-2}, F\textsubscript{1-3}, F\textsubscript{1-3-1}, F\textsubscript{1-3-2}, F\textsubscript{1-3-2-1}, F\textsubscript{1-3-2-2}, F\textsubscript{1-3-3}.



\begin{center}
\begin{adjustbox}{angle=90}
\begin{tabular}{|l|c|c|c|c|}
\hline
\textbf{Caracter\'isticas} &
 \textbf{Soluci\'on 1} &
 \textbf{Soluci\'on 2} &
 \textbf{Soluci\'on 3} &
 \textbf{Soluci\'on 4}\\\hline
C\textsubscript{4-1}.- Forma del mecanismo de agarre &
 Pinzas &
 Ventosas &
 Succionador &
 Pala\\\hline
C\textsubscript{4-2}.- N\'umero de componentes a agarrar &
 1 &
 2 &
 3 &
 4 \'o m\'as\\\hline
\end{tabular}

\end{adjustbox}
\end{center}

Se generaron 16 combinaciones.\par
\par Hay 3,538,944 combinaciones acumuladas.

%%%%%%%%%%%%%%%%%%%%%%%%%%%%%%%%%%%%%%%%%%%%%%%%%%
\textbf{M\'odulo alimentador de componentes}

El m\'odulo alimentador de componentes (M6) pertenece al sistema estructural (S1), las funciones asociadas a este m\'odulo son las siguientes: F\textsubscript{1}
, F\textsubscript{1-1}, F\textsubscript{1-1-1}, F\textsubscript{1-1-2}, F\textsubscript{1-2}, F\textsubscript{1-3}, F\textsubscript{1-3-1}, F\textsubscript{1-3-2}, F\textsubscript{1-3-2-1}, F\textsubscript{1-3-2-2}, F\textsubscript{1-3-3}.


\begin{center}
\begin{adjustbox}{angle=90}
\begin{tabular}{|l|c|c|c|c|}
\hline
\textbf{Caracter\'isticas} &
 \textbf{Soluci\'on 1} &
 \textbf{Soluci\'on 2} &
 \textbf{Soluci\'on 3} &
 \textbf{Soluci\'on 4}\\\hline
C\textsubscript{6-1}.- N\'umero de alimentadores &
 Entre 1 y 3 &
 Entre 4 y 8 &
 Entre 9 y 15 &
 M\'as de 15\\\hline
C\textsubscript{6-2}.- Tipo de motor &
 Motor CD &
 Motor AC &
 Servomotor &
 Motor a pasos\\\hline
C\textsubscript{6-3}.- Componentes a alimentar &
 Componentes sueltos &
 Componentes en su reel-package &
   &
 \\\hline
\end{tabular}

\end{adjustbox}
\end{center}


Se generaron 32 combinaciones.\par
\par Hay 113,246,208 combinaciones acumuladas.\par

%%%%%%%%%%%%%%%%%%%%%%%%%%%%%%%%%%%%%%%%%%%%%%%%%%
\textbf{Interfaz humano-m\'aquina}

La interfaz humano-m\'aquina (M7) pertenece al sistema 0 (S0), las funciones asociadas a este m\'odulo son las siguientes: F\textsubscript{3}, F\textsubscript{3-2}, F\textsubscript{3-3}, F\textsubscript{3-4}, F\textsubscript{3-6}
, F\textsubscript{3-6-1}, F\textsubscript{3-6-2}, F\textsubscript{3-6-2-1}, F\textsubscript{3-6-2-2}.



\begin{center}
\begin{adjustbox}{angle=90}
\begin{tabular}{|l|c|c|c|c|}
\hline
\textbf{Caracter\'isticas} &
 \textbf{Soluci\'on 1} &
 \textbf{Soluci\'on 2} &
 \textbf{Soluci\'on 3} &
 \textbf{Soluci\'on 4}\\\hline
C\textsubscript{7-1}.- Entrada de datos &
 Pantalla t\'actil &
 Interfaz de botones &
 Teclados &
 Mouse\\\hline
C\textsubscript{7-2}.- Retroalimentaci\'on al usuario &
 Interfaz gr\'afica &
 Audio &
 Interfaz de comandos &
 Nada\\\hline
\end{tabular}
\end{adjustbox}
\end{center}

Se generaron 16 combinaciones.\par
\par Hay 1,811,939,328 combinaciones acumuladas.

%%%%%%%%%%%%%%%%%%%%%%%%%%%%%%%%%%%%%%%%%%%%%%%%%%%%%%%%%%%%%%%%%%%%%%%%%%%%%%%%%%%%%%%%%%%%%%%%%%%%
\section{Conceptos soluci\'on}
%
Se presentan los 3 mejores conceptos soluci\'on.
\begin{center}
  \tablefirsthead{%
  \hline
  \multicolumn{2}{|c|}{\textbf{Primer concepto soluci\'on}}\\\hline
  Caracter\'istica & Elecci\'on\\
  \hline}
  \tablehead{%
  \hline
  \multicolumn{2}{|l|}{\small\sl continua de la p\'agina anterior}\\
  \hline
  Caracter\'istica & Elecci\'on\\
  \hline}
  \tabletail{%
  \hline
  \multicolumn{2}{|r|}{\small\sl continua en la siguiente p\'agina}\\
  \hline}
  \tablelasttail{\hline}
  \bottomcaption{Primer concepto soluci\'on}
  
\begin{supertabular}{|c|c|}
\hline

C\textsubscript{1-1} & \\\hline
C\textsubscript{1-2} & \\\hline
C\textsubscript{1-3} & \\\hline
C\textsubscript{1-4} & \\\hline
C\textsubscript{2-1} & \\\hline
C\textsubscript{2-2} & \\\hline
C\textsubscript{2-3} & \\\hline
C\textsubscript{2-4} & \\\hline
C\textsubscript{3-1} & \\\hline
C\textsubscript{3-2} & \\\hline
C\textsubscript{3-3} & \\\hline
C\textsubscript{3-4} & \\\hline
C\textsubscript{4-1} & \\\hline
C\textsubscript{4-2} & \\\hline
C\textsubscript{4-3} & \\\hline
C\textsubscript{4-4} & \\\hline
C\textsubscript{5-1} & \\\hline
C\textsubscript{5-2} & \\\hline
C\textsubscript{5-3} & \\\hline
C\textsubscript{5-4} & \\\hline
C\textsubscript{6-1} & \\\hline
C\textsubscript{6-2} & \\\hline
C\textsubscript{6-3} & \\\hline
C\textsubscript{6-4} & \\\hline
C\textsubscript{7-1} & \\\hline
C\textsubscript{7-2} & \\\hline
C\textsubscript{7-3} & \\\hline
C\textsubscript{7-4} & 0\\
 
\end{supertabular}
\end{center}