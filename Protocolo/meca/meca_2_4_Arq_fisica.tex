\section{Arquitectura f\'isica}
\label{Arquitectura_fisica}

Arquitectura f\'isica
\par
En concordancia con la ''Actividad I.2. Arquitecturas del sistema mecatr\'onico (E)'' se muestra las funciones del sistema, as\'i como la arquitectura f\'isica de nuestro sistema.\par

\begin{itemize}
\item F\textsubscript{0}.- Recoger y Colocar componentes electr\'onicos
\\
\item F\textsubscript{1}.- Manipular la posici\'on de los componentes
\item F\textsubscript{1-1}.- Generar movimiento
\item F\textsubscript{1-1-1}.- Generar movimiento de traslaci\'on
\item F\textsubscript{1-1-2}.- Generar movimiento de rotaci\'on
\item F\textsubscript{1-2}.- Acoplar movimiento
\item F\textsubscript{1-3}.- Mover los componentes
\item F\textsubscript{1-3-1}.- Recoger los componentes en el \'area de trabajo
\item F\textsubscript{1-3-2}.- Orientar los componentes
\item F\textsubscript{1-3-2-1}.- Rotar los componentes
\item F\textsubscript{1-3-2-2}.- Trasladar los componentes
\item F\textsubscript{1-3-3}.- Colocar los componentes en la PCB
\item F\textsubscript{2}.- Gestionar energ\'ia
\item F\textsubscript{2-1}.- Acondicionar energ\'ia el\'ectrica
\item F\textsubscript{2-2}.- Almacenar energ\'ia el\'ectrica
\item F\textsubscript{2-3}.- Distribuir energ\'ia el\'ectrica
\item F\textsubscript{2-4}.- Disipar calor
\\
\item F\textsubscript{3}.- Gestionar informaci\'on
\item F\textsubscript{3-1}.- Medir par\'ametros
\item F\textsubscript{3-1-1}.- Medir par\'ametros internos
\item F\textsubscript{3-1-1-1}.- Medir temperatura interna
\item F\textsubscript{3-1-1-2}.- Medir ubicaci\'on del dispositivo de recolecci\'on
\item F\textsubscript{3-1-1-2-1}.- Medir rotaci\'on del dispositivo de recolecci\'on
\item F\textsubscript{3-1-1-2-2}.- Medir posici\'on del dispositivo de recolecci\'on
\item F\textsubscript{3-1-1-3}.- Medir velocidad del dispositivo de recolecci\'on
\item F\textsubscript{3-1-1-4}.- Medir aceleraci\'on del dispositivo de recolecci\'on
\item F\textsubscript{3-1-1-5}.- Medir fuerza del dispositivo de recolecci\'on
\item F\textsubscript{3-1-2}.- Medir par\'ametros externos
\item F\textsubscript{3-1-2-1}.- Medir temperatura externa
\item F\textsubscript{3-1-2-2}.- Medir ubicaci\'on del dispositivo de componentes
\item F\textsubscript{3-1-2-2-1}.- Medir rotaci\'on del dispositivo de componentes
\item F\textsubscript{3-1-2-2-2}.- Medir posici\'on del dispositivo de componentes
\item F\textsubscript{3-2}.- Acondicionar informaci\'on
\item F\textsubscript{3-3}.- Procesar informaci\'on
\item F\textsubscript{3-4}.- Almacenar informaci\'on
\item F\textsubscript{3-5}.- Tomar desiciones
\item F\textsubscript{3-5-1}.- Diagnosticar fallas
\item F\textsubscript{3-5-2}.- Recuperar funcionalidad
\item F\textsubscript{3-5-3}.- Controlar posici\'on
\item F\textsubscript{3-5-4}.- Controlar velocidad
\item F\textsubscript{3-5-5}.- Controlar aceleraci\'on
\item F\textsubscript{3-5-6}.- Controlar fuerza
\item F\textsubscript{3-5-7}.- Controlar temperatura interna
\item F\textsubscript{3-6}.- Comunicar sistema
\item F\textsubscript{3-6-1}.- Comunicar internamente
\item F\textsubscript{3-6-2}.- Comunicar externamente
\item F\textsubscript{3-6-2-1}.- Interactuar con el usuario
\item F\textsubscript{3-6-2-2}.- Comunicar con otros sistemas
\\
\item F\textsubscript{4}.- Soportar y proteger
\item F\textsubscript{4-1}.- Proteger
\item F\textsubscript{4-1-1}.- Proteger usuario
\item F\textsubscript{4-1-1-1}.- Iluminar \'area de trabajo
\item F\textsubscript{4-1-1-2}.- Indicar l\'imites del \'area de trabajo
\item F\textsubscript{4-1-2}.- Proteger componentes SMD y PCB
\item F\textsubscript{4-1-2-1}.- Inmovilizar PCB
\item F\textsubscript{4-1-2-2}.- Verificar correspondencia de componente con espacio en PCB
\item F\textsubscript{4-1-3}.- Proteger robot
\item F\textsubscript{4-1-3-1}.- Amortiguar vibraciones
\item F\textsubscript{4-1-3-2}.- Impedir sobrecalentamiento
\item F\textsubscript{4-1-3-3}.- Ocultar c\'odigos fuente
\item F\textsubscript{4-1-4}.- Bloquear proceso
\item F\textsubscript{4-2}.- Soportar
\end{itemize}

El objetivo es asociar a cada m\'dulo las funciones en las que son part\'icipes con el objetivo de  proponer caracter\'isticas morfolo\'ogicas a las que se requiere buscar la mejor soluci\'on.

\begin{figure}[hbtp]
\centering
%\includegraphics[scale=0.2]{•}=0,scale=0.16]{images/Arq Fisica.jpg}
\includegraphics[scale=.16]{images/Arq Fisica.jpg} 
\caption{Arquitectura f\'isica del sistema}

\end{figure}

A lo largo del presente documento abordaremos los diferentes m\'odulos del sistema haciendo \'enfasis den las caracter\'isticas que resultan relevantes para considerar en base a estas posibles soluciones que puedan aportar al concepto soluci\'on.