\section{Planeaci\'on}
% 
Para poder planear todas las actividades referentes a la planeaci\'on se desglosa de la siguiente forma:\newline
\begin{itemize}
	\item Ingenier\'ia de Requerimientos: Esta actividad se refiere a deducir los requerimientos con los que debe de cumplir el sistema, todo en base a las necesidades del proyecto.
	\item Especificaciones: En esta actividad se obtienen las especificaciones t\'ecnicas que el sistema deber\'a cumplir, es decir, las restricciones que acotar\'an en universo de soluciones aplicables al sistema.
	\item Dise\~{n}o del sistema: En esta actividad se deber\'a de escoger la mejor soluci\'on que resuelva el dise\~{n}o del sistema.
	\item Modelado y simulaci\'on: Est\'a actividad tiene por objetivo obtener las expresiones matem\'aticas por las cuales est\'an relacionados todos los m\'odulos del sistema.
	\item Dise\~{n}o de componentes: Aqu\'i necesitaremos establecer las caracter\'isticas de cada componente para todos los ensambles de cada m\'odulo que compone el sistema.
	\newline\newline\newline\newline\newline\newline\newline\newline\newline
	\item Prototipo: La idea es poder contemplar todo el sistema para ver cuestiones que puedan ser mejoradas u optimizadas.
	\item Componentes Mecatr\'onicos: Se planea verificar que los componentes est\'en listos para su integraci\'on.
	\item Prueba de componentes: Se pretende realizar verificaciones que permitan saber con certeza que los componentes ser\'an resistentes a cargas.
	\item Sistema de integraci\'on (Hardware): En esta actividad se realizar\'an todos los ensambles requeridos para formar los m\'odulos del sistema.
	\item Sistema de integraci\'on (Software): Se trabajar\'a  sobre todas las cuestiones que tienen que ver con algoritmos de control que permitan el correcto funcionamiento del sistema.
	\item Pruebas del Sistema: Se realizar\'an pruebas que comprueben que el sistema funciona correctamente sin interferencias y que el sistema es seguro.
	\item Pruebas de campo: Implementar el sistema de forma real para comprobar el comportamiento.
	\item Producci\'on: Planear una l\'inea de producci\'on eficiente que se encarga de producir y ensamblar los componentes necesarios del sistema.
\end{itemize}

Aunque algunas de las actividades finales no sean factibles de realizar (pruebas de campo, producci\'on entre otras) es importante tener en cuenta que son actividades que se realizan en un proyecto de un sistema mecatr\'onico.
\newline
Se muestra a continuaci\'on el cronograma generado para el desarrollo del sistema mecatr\'onico durante el resto del semestre.
\newline\newline\newline\newline

%\begin{figure}[hbtp]
%\caption{Cronograma de actividades para m\'aquina Pick\&Place}
%\centering
%\includegraphics[scale=0.15]{images/Cronograma.png}
%\end{figure}