\newpage
\section{Administración del proyecto}
\label{Presupuesto destinado}

\subsection{Presupuesto estimado e infraestructura}
%Recursos humanos, presupuesto e infraestructura\\

\begin{table}[!ht]
	\centering
	\caption{Recursos humanos disponibles para el desarrollo del proyecto.}
	\begin{tabular}{ccc}
		\toprule
		\textbf{Recursos humanos} & \textbf{Institución}  & \textbf{Tiempo destinado} \\ \midrule
		        Alumno 1          &      IPN-UPIITA       &         820 horas         \\
		        Alumno 2          &      IPN-UPIITA       &         820 horas         \\ \midrule
		        Asesor 1          &      IPN-UPIITA       &         100 horas         \\
		        Asesor 2          &      IPN-UPIITA       &         100 horas         \\ \midrule
		                          & \textbf{Tiempo total} &        1800 horas         \\ \bottomrule
	\end{tabular}
\end{table}

Analizando a los módulos y subsistemas anteriores se hace una estimación de los recursos a utilizar
\begin{table}[!ht]
	\centering
	\caption{Presupuesto estimado para el desarrollo del proyecto.}
	\begin{tabular}{cccl}
		\toprule
		            \textbf{Módulo}              & \textbf{Recurso/Material} &  \textbf{Cantidad}   & \textbf{Costo} \\ \midrule
		\multirow{2}{*}{Módulo de procesamiento} &           FPGA            &          1           & \$3500         \\
		                                         &    Cables y conectores    &        Varios        & \$500          \\ \midrule
		\multirow{2}{*}{Módulo de procesamiento} &           FPGA            &          1           & \$3500         \\
		                                         &    Cables y conectores    &        Varios        & \$500          \\ \midrule
		\multirow{2}{*}{Módulo de procesamiento} &           FPGA            &          1           & \$3500         \\
		                                         &    Cables y conectores    &        Varios        & \$500          \\ \midrule
		                                         &                           & \textbf{Costo total} & \$ 12000       \\ \bottomrule
	\end{tabular}
\end{table}

Para el desarrollo del proyecto se planea utilizar la siguiente infraestructura
\begin{table}[!ht]
	\centering
	\caption{Infraestructura para el desarrollo del proyecto.}
	\begin{tabular}{c c c c }
		\toprule
		              \textbf{Infraestructura}                & \textbf{Equipo}          & \textbf{Cantidad} & \textbf{Uso}                                             \\ \midrule
		   \multirow{3}{25mm}{Laboratorio de electrónica}     & Fuente de CC             & 2                 & \multirow{3}{40mm}{Verificar los circuitos electrónicos} \\
		                                                      & Osciloscopio             & 1                 &                                                          \\
		                                                      & Generador de funciones   & 1                 &                                                          \\ \midrule
		       \multirow{3}{20mm}{Taller de máquinas}         & Fresadora                & 1                 & \multirow{3}{40mm}{Maquinar piezas del sistema}          \\
		                                                      & Centro de mecanizado CNC & 1                 &                                                          \\
		                                                      & Vernier                  & 1                 &                                                          \\ \midrule
		\multirow{3}{25mm}{Laboratorio de sistemas digitales} & Máquina CNC              & 1                 & \multirow{3}{40mm}{Fabricar el circuito impreso}         \\
		                                                      & Fresadora                & 2                 &                                                          \\
		                                                      & Broca                    & 6                 &                                                          \\ \bottomrule
	\end{tabular}
\end{table}

\subsection{Planeación de actividades}
La planeación de las actividades se realiza en función de los módulos deñl sistema

%\newpage
\subsubsection{Cronograma de actividades correspondientes a Trabajo Terminal I}

XD

\begin{table}[!ht]
	\caption{Cronograma de TT1}
	\resizebox{\textwidth}{!}{%
		\begin{tabular}{clccccccccccccccccccc}
			\toprule
			\multirow{2}{*}{\#} & \multirow{2}{*}{Actividad}                                   &             &                                             \multicolumn{18}{c}{Semanas}                                             \\
			                    &                                                              & Responsable &          1           &          2           & 3 & 4 & 5 & 6 & 7 & 8 & 9 & 10 & 11 & 12 & 13 & 14 & 15 & 16 & 17 & 18 \\ \midrule
			    \textbf{1}      & \textbf{Diseño del sistema}                                  &             &                      &                      &   &   &   &   &   &   &   &    &    &    &    &    &    &    &    &    \\
			        1.1         & Evaluación de necesidades                                    & $U\omega U$ & \cellcolor{colorIPN} &                      &   &   &   &   &   &   &   &    &    &    &    &    &    &    &    &    \\
			        1.2         & Definición de requerimientos                                 & $U\omega U$ & \cellcolor{colorIPN} &                      &   &   &   &   &   &   &   &    &    &    &    &    &    &    &    &    \\
			        1.3         & Definición de funciones                                      & $U\omega U$ &                      & \cellcolor{colorIPN} &   &   &   &   &   &   &   &    &    &    &    &    &    &    &    &    \\
			        1.4         & Definición de módulo y sistemas                              & $U\omega U$ &                      & \cellcolor{colorIPN} &   &   &   &   &   &   &   &    &    &    &    &    &    &    &    &    \\
			        1.5         & Generación de propuestas                                     & $U\omega U$ &                      & \cellcolor{colorIPN} &   &   &   &   &   &   &   &    &    &    &    &    &    &    &    &    \\
			        1.6         & Establecimiento de un concepto solución multi-dominio        & $U\omega U$ &                      & \cellcolor{colorIPN} &   &   &   &   &   &   &   &    &    &    &    &    &    &    &    &    \\ \midrule
			    \textbf{2}      & \textbf{Diseño del dominio específico}                       &             &                      &                      &   &   &   &   &   &   &   &    &    &    &    &    &    &    &    &    \\
			        2.1         & Diseño del sistema de orientación                            &             &                      &                      &   &   &   &   &   &   &   &    &    &    &    &    &    &    &    &    \\
			        2.2         & Diseño del módulo de control y procesamiento                 &             &                      &                      &   &   &   &   &   &   &   &    &    &    &    &    &    &    &    &    \\ \midrule
			    \textbf{3}      & \textbf{Integración computacional de los módulos y sistemas} &             &                      &                      &   &   &   &   &   &   &   &    &    &    &    &    &    &    &    &    \\
			        3.1         & Diseño del sistema de orientación                            &             &                      &                      &   &   &   &   &   &   &   &    &    &    &    &    &    &    &    &    \\
			        3.2         & Diseño del módulo de control y procesamiento                 &             &                      &                      &   &   &   &   &   &   &   &    &    &    &    &    &    &    &    &    \\ \midrule
			    \textbf{4}      & \textbf{Diseño de ...}                                       &             &                      &                      &   &   &   &   &   &   &   &    &    &    &    &    &    &    &    &    \\
			        4.1         & Diseño del sistema de orientación                            &             &                      &                      &   &   &   &   &   &   &   &    &    &    &    &    &    &    &    &    \\
			        4.2         & Diseño del módulo de control y procesamiento                 &             &                      &                      &   &   &   &   &   &   &   &    &    &    &    &    &    &    &    &    \\ \bottomrule
		\end{tabular}
	}
\end{table}

%\newpage
\subsubsection{Cronograma de actividades correspondientes a Trabajo Terminal II}
XD
\begin{table}[!ht]
	\caption{Cronograma de TT2}
	\resizebox{\textwidth}{!}{%
		\begin{tabular}{clccccccccccccccccccc}
			\toprule
			\multirow{2}{*}{\#} & \multirow{2}{*}{Actividad} &  & \multicolumn{18}{c}{Semanas}\\
			&  & Responsable & 1 & 2 & 3 & 4 & 5 & 6 & 7 & 8 & 9 & 10 & 11 & 12 & 13 & 14 & 15 & 16 & 17 & 18 \\
			\midrule
			\textbf{1} & \textbf{Manufactura e implementación del módulo de energía} &  &  &  &  &  &  &  &  &  &  &  &  &  &  &  &  &  &  &  \\
			1.1 & Evaluación de necesidades & $U\omega U$ & \cellcolor{colorIPN} &  &  &  &  &  &  &  &  &  &  &  &  &  &  &  &  &  \\
			1.2 & Definición de requerimientos & $U\omega U$ & \cellcolor{colorIPN} &  &  &  &  &  &  &  &  &  &  &  &  &  &  &  &  &  \\
			1.3 & Definición de funciones & $U\omega U$ &  & \cellcolor{colorIPN} &  &  &  &  &  &  &  &  &  &  &  &  &  &  &  &  \\
			1.4 & Definición de módulo y sistemas & $U\omega U$ &  & \cellcolor{colorIPN} &  &  &  &  &  &  &  &  &  &  &  &  &  &  &  &  \\
			1.5 & Establecimiento de un concepto solución multi-dominio & $U\omega U$ &  & \cellcolor{colorIPN} &  &  &  &  &  &  &  &  &  &   &  &  &  &  &  &  \\
			\midrule
			\textbf{2} & \textbf{Manufactura e implementación del módulo de control y procesamiento} &  &  &  &  &  &  &  &  &  &  &  &  &  &  &  &  &  &  &  \\
			2.1 & Diseño del sistema de orientación &  &  &  &  &  &  &  &  &  &  &  &  &  &  &  &  &  &  &  \\
			2.2 & Diseño del módulo de control y procesamiento &  &  &  &  &  &  &  &  &  &  &  &  &  &  &  &  &  &  &  \\
			\midrule
			\textbf{3} & \textbf{Manufactura e implementación del módulo de orientación} &  &  &  &  &  &  &  &  &  &  &  &  &  &  &  &  &  &  &  \\
			3.1 & Diseño del sistema de orientación &  &  &  &  &  &  &  &  &  &  &  &  &  &  &  &  &  &  &  \\
			3.2 & Diseño del módulo de control y procesamiento &  &  &  &  &  &  &  &  &  &  &  &  &  &  &  &  &  &  &  \\
			\midrule
			\textbf{4} & \textbf{Manufactura e implementación del módulo de interfaz humano-máquina} &  &  &  &  &  &  &  &  &  &  &  &  &  &  &  &  &  &  &  \\
			4.1 & Diseño del sistema de orientación &  &  &  &  &  &  &  &  &  &  &  &  &  &  &  &  &  &  &  \\
			4.2 & Diseño del módulo de control y procesamiento &  &  &  &  &  &  &  &  &  &  &  &  &  &  &  &  &  &  &  \\
			\midrule
			\textbf{5} & \textbf{Integración computacional de los módulos y sistemas} &  &  &  &  &  &  &  &  &  &  &  &  &  &  &  &  &  &  &  \\
			3.1 & Diseño del sistema de orientación &  &  &  &  &  &  &  &  &  &  &  &  &  &  &  &  &  &  &  \\
			3.2 & Diseño del módulo de control y procesamiento &  &  &  &  &  &  &  &  &  &  &  &  &  &  &  &  &  &  &  \\
			\midrule
			\textbf{6} & \textbf{Realización de pruebas de funcionamiento} &  &  &  &  &  &  &  &  &  &  &  &  &  &  &  &  &  &  &  \\
			4.1 & Diseño del sistema de orientación &  &  &  &  &  &  &  &  &  &  &  &  &  &  &  &  &  &  &  \\
			4.2 & Diseño del módulo de cobntrol y procesamiento &  &  &  &  &  &  &  &  &  &  &  &  &  &  &  &  &  &  &  \\ \bottomrule
		\end{tabular}
	}
\end{table}

